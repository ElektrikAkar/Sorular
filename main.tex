    \documentclass{article}
\usepackage[utf8]{inputenc}
\usepackage{amsmath}
\usepackage{amssymb}
\usepackage{mathtools}
\usepackage[shortlabels]{enumitem}
\usepackage[right]{eurosym} %for euro sign
\usepackage[ddmmyyyy]{datetime}
\usepackage{wrapfig}
\usepackage{braket}

\usepackage{url,parskip} % Formatting packages

\usepackage[usenames,dvipsnames]{xcolor} % Required for specifying custom colors



\usepackage{hyperref} % Required for adding links	and customizing them
\definecolor{linkcolour}{rgb}{0.03,0.459,0.718} % Link color
\definecolor{linkcolour2}{rgb}{0.6,0.0,0.0} % Link color
\definecolor{linkcolour3}{rgb}{0,0.1,0.9} % Link color
\definecolor{light_color}{rgb}{0.2,0.2,0.2} % Link color
\newcommand{\light}[1]{\textcolor{light_color}{#1}}
\newcommand{\VkWebsiteUrl}[2]{\href{#1}{\textcolor{linkcolour2}{#2}}}
\newcommand{\VkcitationUrl}[2]{\href{#1}{\textcolor{linkcolour3}{#2}}}
\numberwithin{equation}{section}
\title{Sorular ( \VkWebsiteUrl{mobile.html}{Mobil sürüm için tıklayın} )}
\author{VK}

\begin{document}


\maketitle


Bazen çevreden hoşuma giden sorular görüyorum. Bunları da çözüp atmak yerine bir yerde depolayıp örnek sorular oluşturma fikri aklıma geldi. O yüzden Github'ta tutayım dedim, belki gün gelir yeterli soru birikirse millet de faydalanır (faydalanamadı).

Not: Bazı çözümler hatalıdır.

\section{Soru: 2020/06/02} 
$$\oint\limits_{\mathcal{C}=1} \frac{\sin(z)}{z^4}\,\mathrm{d}z \; \text{ where } \mathcal{C} =\{z: |z|=1, 0\leqslant \arg(z) \leqslant 2\pi \}$$

\textbf{Cevap: }

Aslında bu sorunun çözümü kolayca \VkWebsiteUrl{https://math.stackexchange.com/questions/648066/evaluate-the-contour-integral-int-gamma0-1-frac-sinzz4dz}{Cauchy Integrali} üzerinden bulunabilir. Ancak \VkcitationUrl{https://twitter.com/giritli_dogrucu/status/1266107079011569664}{@giritli\_dogrucu} tarafından çözüm ilgimi çektiği için onun yolu ile çözmek istiyorum. İlk aşamada onun geldiği yere kadar olan kısmı ondan alıntı yaparak buraya yazıyorum. 

$|z| =1$ olduğu için $z = e^{i\theta}$ burada $\theta \in [0, 2\pi]$  yani birim çember (genel olarak  $z = re^{i\theta}$ yazılabilir). Integralde değişken dönüşümü yapmak için 
\begin{align}
	\frac{\mathrm{d}z}{\mathrm{d}\theta} &= i e^{i\theta} \text{ ise } \mathrm{d}z = i e^{i\theta} \mathrm{d}\theta\\
	\label{eq:20200602:intbase}
	I &= \oint\limits_{\mathcal{C}=1} \frac{\sin(z)}{z^4}\,\mathrm{d}z = \int_0^{2\pi} \frac{\sin(e^{i\theta})}{e^{4i\theta}} \cdot i e^{i\theta} \mathrm{d}\theta \nonumber \\
	&= i\int_0^{2\pi} \frac{\sin(e^{i\theta})}{e^{3i\theta}}  \mathrm{d}\theta
\end{align}

Çözüm \VkcitationUrl{https://twitter.com/giritli_dogrucu/status/1266107079011569664}{@giritli\_dogrucu} tarafından buraya kadar yapılmış olup, geriye kalan kısım egzersiz olması amacıyla öğrenciye bırakılmıştır. Ben de egzersiz yapmayı seven biri olarak çözümün kalanını için de uğraşmak istedim. Şöyle ki herkes seriye açarak integral alıyor, ben niye almayayım değil mi :) (alamadı).

Bu soruyu çözmek için öncelikle sinüs fonksiyonunun Taylor serisine açılımını anımsayalım Denklem \eqref{eq:sinus_taylor}.
%
\begin{align}
\label{eq:sinus_taylor}
	\sin(x) = \sum_{n=0}^{\infty} (-1)^n \frac{x^{2n+1}}{(2n+1)!}
\end{align}

Denklem \eqref{eq:sinus_taylor}'i kullanarak Denklem \eqref{eq:20200602:intbase} içerisindeki sinüsü seriye açarsak: 
%
%\sin(e^{i\theta})
\begin{align}
 I = i\int_0^{2\pi} \frac{\sum_{n=0}^{\infty} (-1)^n \frac{(e^{i\theta} )^{2n+1}}{(2n+1)!}}{e^{3i\theta}}  \mathrm{d}\theta
\end{align} 

Burada integral ile toplam sembolunun yerini değiştirip $\theta$'ya bağlı olmayan terimleri dışarı atarsak: 
\begin{align}
I = i \sum_{n=0}^{\infty} \frac{(-1)^n}{(2n+1)!}  \int_0^{2\pi} \frac{ e^{i\theta\cdot (2n+1)} }{e^{3i\theta}}  \mathrm{d}\theta
\end{align} 

Integralin içerisindeki pay ve paydayı tabanlar aynı olduğu için birleştirebiliriz.

\begin{align}
I = i \sum_{n=0}^{\infty} \frac{(-1)^n}{(2n+1)!}  \int_0^{2\pi} e^{i\theta\cdot (2n-2)}  \mathrm{d}\theta
\end{align} 

Artık intagralimiz öcü formundan çıkıp herkesin alabileceği bir şekle büründü (bu sefer de toplam sembolü öcü gibi duruyor, bunu tekrar nasıl fonksiyona dönüştüreceğiz :O). Öyleyse integrali alalım. (Not: toplam sembolünün solundaki $i$ ile paydadaki $i$ sadeleşiyor).

\begin{align}
I &= i\sum_{n=0}^{\infty} \frac{(-1)^n}{(2n+1)!} \left( \frac{ e^{i\theta\cdot (2n-2)}  }{i\cdot (2n-2)}\right|_0^{2\pi} \nonumber \\
&= \sum_{n=0}^{\infty} \frac{(-1)^n}{(2n+1)!} \left( \frac{ e^{i2\pi\cdot (2n-2)} -  e^{i\cdot  0\cdot (2n-2)}  }{(2n-2)}\right) \nonumber \\
&= \sum_{n=0}^{\infty} \frac{(-1)^n}{(2n+1)!} \left( \frac{ e^{i2\pi\cdot (2n-2)} -  1  }{(2n-2)}\right)
\end{align} 

Tamam bir şeyler çıktı peki bu toplam sembolünü nasıl hesaplayacağız? 1 değil 2 değil sonsuza kadar gidiyor. Fark ettiyseniz Değerleri yerine yazarken 0'ı yerine yazdık ancak $2\pi$'yi neredeyse sıfır gibi değerlendirilebileceği halde yerine yazmadık. Nasıl olsa üst $2\pi$'nin tam katı çıktığı sürece bu kısmın da sıfır olması lazım değil mi? Evet ama $n=1$ için hem pay hem payda sıfır olduğu için belirsizlik durumu söz konusu. Yani toplam sembolümüzün içindeki tüm terimler $n\neq 1$ için sıfır veriyor. Ancak $n=1$ için de belirsizlik veriyor. Yani 0'ları da toplamak çok kolay olduğu için toplam sembolümüz tek terime iner. 
\begin{align}
I&= \lim_{n \to 1} \frac{(-1)^n}{(2n+1)!} \left( \frac{ e^{i2\pi\cdot (2n-2)} -  1  }{ 2n-2}\right)
\end{align} 

Evet gördüğünüz üzere toplam sembolünden de kurtulduk (bu sefer de limit geldi yav). Peki bu limit ile ne yapacağız? Tabi ki lisede öğrendiğimiz gibi L' Hospital kuralını işleteceğiz yani payın ve paydanın türevini alacağız (yoksa bunu da mı seriye açsaydım). Ancak faktöriyelin türevini almayı pek seven biri olmadığım için limiti belirsizlik içerenler ve içermeyenler olarak ikiye ayıracağım. Şöyle ki: 
\begin{align}
I&= \left(\lim_{n \to 1} \frac{(-1)^n}{(2n+1)!} \right) \left(\lim_{n \to 1}\left( \frac{ e^{i2\pi\cdot (2n-2)} -  1  }{2n-2}\right)\right) \nonumber \\
&= \frac{-1}{3!} \cdot \lim_{n \to 1}\left( \frac{ e^{i2\pi\cdot (2n-2)} -  1  }{2n-2}\right)
\end{align} 

Şimdi türev alalım: 
\begin{align}
I&= \frac{-1}{3!} \cdot \lim_{n \to 1}\left( \frac{ 4i\pi\cdot e^{i2\pi\cdot (2n-2)}  }{2}\right)
\end{align} 

Artık belirsizlik kaybolduğu için $n=1$ yerine koyalım ve sonucu bulalım. 

\begin{align}
I&= \frac{-1}{3!} \cdot \left( \frac{ 4i\pi\cdot 1}{2}\right) \nonumber \\
&= -\frac{2i\pi}{3!}  \nonumber \\ 
&= -\frac{i\pi}{3}
\end{align}

olarak Cauchy Integrali ile aynı sonuç bulunur. Fark edildiyse, sadece belirsizlik oluşturan kısım toplam sembolünde fark oluşturdu, bu da Cauchy Integralindeki sadece çemberin içinde kalan tekil noktaların etkisi hakkında bir fikir veriyor. 

\section{Soru: 2020/05/03}
$F_2(x)$ üzerinde $\braket{p(x),q(x)} = \int_{0}^{1}p(x)q(x)\mathrm{d}x$ iç çarpımına göre $F_2(x)$'in $S=\{x^2,x\}$ tabanina sahip $W$ alt uzayı için ortonormal bir taban bulunuz. 

\textbf{Cevap: }

Bu soruyu çözmek için Gram-Schmidt methodunu kullanabiliriz. Çok fark etmemekle birlikte ilk vektörümüzü $s_1 = x^2$ yönündeki birim vektör olarak seçersek: 
%
\begin{align}
	u_1^\prime &= x^2 \\
	u_1 &= \frac{u_1^\prime}{|u_1^\prime|} \nonumber\\
	&= \frac{x^2}{\sqrt{\braket{x^2,x^2}}}\nonumber \\
	&= \frac{x^2}{\sqrt{\int_{0}^{1}(x^2\cdot x^2 \mathrm{d}x)}} \nonumber \\
	&= \frac{x^2}{\sqrt{1/5}} = \sqrt{5}x^2 
\end{align}

Daha sonra ise bulduğumuz $s_2$'nin $u_1$ yönündeki izdüşümü $s_2 = x$'ten çıkartılarak, ikinci dik vektör elde edilir. 
%
\begin{align}
u_2^\prime &= s_2 - \braket{s_2,u_1} u_1 \nonumber \\
&= x - \braket{x, \sqrt{5}x^2} \sqrt{5}x^2 \nonumber \\
&= x - \sqrt{\int_{0}^{1}x\cdot \sqrt{5}x^2 \mathrm{d}x} \; \sqrt{5}x^2 \nonumber \\
&= x - \sqrt{5/16} \; \sqrt{5}x^2 \nonumber \\
&= x - \frac{5x^2}{4} 
\end{align}
%
Bu vektör de boyuna bölünerek ortonormal tabanın 2. birim vektörü bulunmuş olur.
%
\begin{align}
u_2 &= \frac{u_2^\prime}{|u_2^\prime|} \nonumber\\
&= \frac{x - 5x^2/4 }{\sqrt{\braket{x - 5x^2/4,x - 5x^2/4}}}\nonumber \\
&= \frac{x - 5x^2/4}{\sqrt{\int_{0}^{1}((x - 5x^2/4)\cdot (x - 5x^2/4) \mathrm{d}x)}} \nonumber \\
&= \frac{x - 5x^2/4}{\sqrt{3}/12} = 4\sqrt{3}x- 5\sqrt{3}x^2
\end{align}

Böylelikle ortonormal taban $\{\sqrt{5}x^2,\; 4\sqrt{3}x- 5\sqrt{3}x^2\}$ olarak bulunur. İşlemin doğruluğunu kontrol etmek için aşağıdaki işlemler yapılabilir. 
%
\begin{align}
\braket{u_1,u_2} &= \int_{0}^{1}\left(\sqrt{5}x^2 \cdot ( 4\sqrt{3}x- 5\sqrt{3}x^2)\right)\mathrm{d}x = 0 \\ 
\braket{u_1,u_1} &= \int_{0}^{1}\left(\sqrt{5}x^2 \cdot \sqrt{5}x^2\right)\mathrm{d}x = 1 \\ 
\braket{u_2,u_2} &= \int_{0}^{1}\left(( 4\sqrt{3}x- 5\sqrt{3}x^2) \cdot ( 4\sqrt{3}x- 5\sqrt{3}x^2)\right)\mathrm{d}x = 1 
\end{align}
 
 

\section{Soru: 2020/04/04}
\textbf{Soru:}
$$
\left. \begin{array}{r}
x-ay +z = 1\\
ax - y +z =a \\
x + y -z = 0 
\end{array} \right\} \text{lineer denklem sisteminin}
$$

\begin{enumerate}[(a)]
	\item tek çözümünün olması için $a=$ ?
	\item çözümünün olmaması için $a=$ ?
	\item sonsuz çözümünün olması için $a=$ ?
\end{enumerate}
\textbf{Cevap: }
Denklemi öncelikle matris formunda yazalım. 
%
\begin{align}
	\underbrace{\left[\begin{matrix}
		1 & -a & 1 \\
		a & -1 & 1 \\
		1 &  1 & -1 
		\end{matrix}\right]}_{A}\left[\begin{matrix}
	x \\
	y \\
	z
	\end{matrix}\right] = \underbrace{\left[\begin{matrix}
		1 \\
		a \\
		0
		\end{matrix}\right]}_{b}
\end{align}

sol ve sağ tarafı birleştirerek arttırılmış matrisi $(A|b)$ oluşturalım. 
%
\begin{align}
\left[\begin{array}{ccc|c}
1 & -a & 1  &1 \\
a & -1 & 1  &a\\
1 &  1 & -1 &0
\end{array}\right]
\end{align}

Gauss eleme yöntemini uygulamak için 3. satırı $-1$ ile çarpıp 1. satıra, $-a$ ile çarpıp 2. satıra eklersek: 
%
\begin{align}
\left[\begin{array}{ccc|c}
0 & -a-1 & 2  & 1 \\
0 & -a-1   & a+1  & a\\
1 &  1   & -1 & 0
\end{array}\right]
\end{align}

2. satırı -1 ile çarpıp 1. satıra eklersek
%
\begin{align}
\left[\begin{array}{ccc|c}
0 & 0      & 1-a  & 1-a \\
0 & -a-1   & a+1  & a\\
1 &  1   & -1 & 0
\end{array}\right]
\end{align}
%

katsayılar kısmının determinantına, 3. satır 1. sütun elemanına göre Laplace açılımıyla baktığımızda: 

\begin{align}
\det(A) = 1\cdot (1-a)\cdot (1+a) = 1 - a^2
\end{align}

\textbf{a)} Tek çözümü olması için $\det(A)\neq 0$ olmalı. Bu yüzden $a \in \mathbb{R} \backslash \{1,-1\}$ \\

\textbf{b)} Sonsuz çözüm için ise arttırılmış matrisin rank değerinin satır sayısından küçük olması gerekir. Katsayılar kısmının determinantının $a \in \{1,-1\}$ olduğunu biliyorsak. $a=1$ için $(A|b)$'nin ilk satırının tamamen 0 olduğu açıkça görülmektedir. Böylelikle $\text{rank}((A|b)) = 2<3$ olacaktır. Bu durumda sonsuz çözüm elde edilir.  \\

\textbf{c)} Hiç çözümün olmaması için ise katsayılar matrisinin rank değeri 3'ten küçük olduğu halde arttırılmış matrisin rank değerinin 3 olması gereklidir. Bu durum da  $a=-1$ için sağlanır. \\

Bu yaptığımız işlemlerin doğruluğunu gözle görmek için üstteki şartları denklem takımımıza yerleştirirsek. $a=1$ için:
$$
\begin{array}{l}
x - y +z = 1\\
x - y +z = 1 \\
x + y -z = 0 
\end{array} 
$$

Görüldüğü üzere 1. ve 2. denklem aynı olduğundan dolayı ikisinden biri silinerek gerçekte 2 tane denklem 3 tane değişken içeren bir sistemimiz olduğu görülür. Böylelikle 3. değişken bağımsız değişken olup, bu değişken sonsuz farklı değer alabileceğinden, denklem sistemimizin sonsuz çözümü vardır. Bundan farklı olarak, yine katsayılar matrisinin determinantını sıfır yapıyor olsa da $a=-1$ durumu için denklem sistemini incelersek:
%
$$
\begin{array}{ll}
x + y + z &= 1\\
-x - y + z &=-1 \\
x + y -z &= 0 
\end{array}
$$

Burada da görüldüğü gibi, 2. denklemin sol tarafını -1 ile  çarptığımızda 3. denklemin sol tarafını elde edebiliyor olmamıza rağmen, sağ tarafı 1 olmaktadır. Bu da 3. denklemin 0 olan sağ tarafından farklıdır. Yani 2. ve 3. denklemlerin birbiri ile çeliştiği bellidir. Bu sebepten ötürü çözüm kümesi boştur. \\

Son olarak da grafik bir anlam yüklemek istersek, üstte verilen denklem sistemleri 3 boyutlu uzayda çeşitli düzlemleri ifade eder. Bir denklem sisteminin çözümünü kümesi ise bu düzlemlerin kesişiminde kalan noktalar kümesi olarak düşünülebilir. Katsayılar matrisinin determinantının sıfır olması, düzlemlerin birbirine paralel olduğuna işarettir. Denklemin sağ tarafındaki sayıların farkı da bu iki düzlem arasındaki uzaklığa işaret ettiğinden, sağ tarafı aynı olan iki paralel düzlem, aynı düzlemdir ve tüm noktalarda kesişir. Ancak eğer sağ tarafları farklıysa, aralarında sabit bir uzaklık bulunan bu iki düzlem, asla kesişmezler ve böylelikle sonuç da bulunamaz. Bunun belki grafiğini de çizerim ileride (çizmedi).
 






\section{Soru: 2020/03/12}
\textbf{Soru:} $\int_0^\infty te^{-kt^2}\mathrm{d}t = \frac{1}{2k}$ ise $n$'in tek degerleri icin $\int_0^\infty t^ne^{-kt^2}\mathrm{d}t$ nedir?
\\ \\
\textbf{Cevap: }\\

$n$ tek ise $n = 2m+1 \quad \forall m \in \{0,1,2,\cdots \}$ yazılabilir. Öyleyse integral: 

\begin{equation}
    \int_0^\infty t^{2m+1}e^{-kt^2}\mathrm{d}t = \int_0^\infty t^{2m}te^{-kt^2}\mathrm{d}t
\end{equation}

Bunu integral reduction ile çözmeliyiz. Ama önce değişken dönüşümü yapalım. $z = t^2$ için $\mathrm{d}z = 2t\mathrm{d}t$ olur.
\begin{align}
    z &= t^2 \\
    \mathrm{d}z&= 2t\mathrm{d}t \\
    t^{2m} &= (t^2)^m = z^m \\
    t &\rightarrow 0, \quad z \rightarrow 0 \\
    t &\rightarrow \infty, \quad  z \rightarrow \infty
\end{align}

Öyleyse:

\begin{equation}
     \underbrace{\frac{1}{2} \int_0^\infty z^m e^{-kz}\mathrm{d}z}_{I(m)}
\end{equation}

Bundan sonra $I(m)$'i çözmek integral reduction ile çok kolay! Tabi yardımımıza kısmi integrasyon yetişsin. $\int vdu = uv - \int udv$
\begin{align}
    v &= z^m \\
    dv&= mz^{m-1} \\
    du&= e^{-kz}\mathrm{d}z \\
    u &= -e^{-kz}/k 
\end{align}

Yerlerine yazarsak:

\begin{align}
    \label{eq:Im}
    I(m) = \frac{1}{2}\left(\underbrace{-e^{-kz}z^m/k}_{f(z)}\right|_0^\infty + \frac{m}{k}\underbrace{\frac{1}{2}\int_0^\infty z^{m-1}e^{-kz}\mathrm{d}z}_{I(m-1)} 
\end{align}

Burada $f(z)$ hesaplanırsa 
\begin{align}
\lim_{z \to \infty} f(z) &= \lim_{z \to \infty} \left(-\frac{z^m}{e^{kz}}\right) = 0 \\
f(0) &= 0 
\end{align}


Böylelikle Denklem \eqref{eq:Im}: 
\begin{align}
    I(m) = \frac{m}{k} I(m-1) 
\end{align}

$m=1$ için soruda verildiği üzere $I(0) = 1/(2k)$ olduğundan: 

\begin{align}
    I(1) &= \frac{1}{k} \cdot \frac{1}{2k} = \frac{1}{2k^2} \\
    I(2) &= \frac{2}{k} \cdot \frac{1}{2k^2} = \frac{1}{k^3} \\
    I(3) &= \frac{3}{k} \cdot \frac{1}{k^3} = \frac{3}{k^4} \\
      & \;\;\vdots \\
    I(m) &= \frac{m!}{2k^{m+1}} 
\end{align}

Burada tabi soruda $n$ verildigi icin $m$ cinsinden birakmak yakışık almaz. Onu da çevirmek lazım (çevirmedi)




\end{document}
